\section{Glossary}
{\renewcommand{\arraystretch}{0.5}
 \begin{longtable}{ R{0.15\linewidth}  L{0.81\linewidth}  }

\rowlabel{glossary:active} \\* \textbf{active} & \parbox[t]{\linewidth}{This term may depend mulitple things depending on the context. A flag is said to be active if its activity has been set as such in a scenario. A time-slice is said to be active in every turn in which the players are prompted to command their stones in that time-slice. See \hyperref[glossary:activity]{\textbf{activity}}, \hyperref[glossary:turn]{\textbf{turn}}.}\\
\rowlabel{glossary:activity} \\* \textbf{activity} & \parbox[t]{\linewidth}{A flag can be set as active or passive. An active flag behaves normally. A passive flag is ignored by the stone it's attached to, and the stone behaves as if the flag was not placed at all. See \hyperref[glossary:flag]{\textbf{flag}}, \hyperref[glossary:active]{\textbf{active}}.}\\
\rowlabel{glossary:allegiance} \\* \textbf{allegiance} & \parbox[t]{\linewidth}{A property of a base, determining the last faction which conquered that base, or "neutral" if determined as such on setup. See \hyperref[glossary:faction]{\textbf{faction}}.}\\
\rowlabel{glossary:ante-effect} \\* \textbf{ante-effect} & \parbox[t]{\linewidth}{An effect that precedes its cause on the time-axis. See \hyperref[glossary:retro-cause]{\textbf{retro-cause}}.}\\
\rowlabel{glossary:azimuth} \\* \textbf{azimuth} & \parbox[t]{\linewidth}{The orientation of an orientable stone, which can be up (negative $y$ direction), right (positive $x$ direction), down (positive $y$ direction), or left (negative $x$ direction). See \hyperref[glossary:orientable]{\textbf{orientable}}.}\\
\rowlabel{glossary:base} \\* \textbf{base} & \parbox[t]{\linewidth}{A square which can be captured by a player visiting it with one of their stones, and it remains captured until their opponent visits it themselves. The goal of the game is to reach a scenario where all the bases end up captured by your stones in the final time-slice. See \hyperref[glossary:allegiance]{\textbf{allegiance}}.}\\
\rowlabel{glossary:canon} \\* \textbf{canon} & \parbox[t]{\linewidth}{The canon of the board is its evolution throughout the turns of the game. Even though time may go back, turns always progress forward, and once the state of the board for a specific turn has been determined, it cannot change again--we say it is canonical. A canonical state cannot be conflicting. See \hyperref[glossary:conflicting]{\textbf{conflicting}}.}\\
\rowlabel{glossary:canonisation} \\* \textbf{canonisation} & \parbox[t]{\linewidth}{At the end of each round, the highest-priority causally consistent scenario is selected for the next round based on the flags placed so far, omitting ante-effects placed in the round which just ended. See \hyperref[glossary:causally consistent]{\textbf{causally consistent}}.}\\
\rowlabel{glossary:causally consistent} \\* \textbf{causally consistent} & \parbox[t]{\linewidth}{Not resulting in a paradox. See \hyperref[glossary:scenario]{\textbf{scenario}}, \hyperref[glossary:paradox]{\textbf{paradox}}.}\\
\rowlabel{glossary:causally free} \\* \textbf{causally free} & \parbox[t]{\linewidth}{A stone is causally free at a specific position if it is placed on the board at that position, but is not subject to a flag at that position.}\\
\rowlabel{glossary:causally locked} \\* \textbf{causally locked} & \parbox[t]{\linewidth}{Present on the board, but not causally free. See \hyperref[glossary:causally free]{\textbf{causally free}}.}\\
\rowlabel{glossary:conflicting} \\* \textbf{conflicting} & \parbox[t]{\linewidth}{Not adhering to the rules which specify what the board can and cannot look like. Typically, the movement of stones in each turn sends them into conflicting positions, which are subsequently resolved by the game to find the canon state of the next time-slice. See \hyperref[glossary:canon]{\textbf{canon}}.}\\
\rowlabel{glossary:dogma} \\* \textbf{dogma} & \parbox[t]{\linewidth}{A flag is said to be dogmatic in a specific round if it is set as active in the round's scenario due to a special rule (the \hyperref[sec:rule of dogma]{rule of dogma}), not because it was set as active due to the selection of a high-priority causally consistent scenario. See \hyperref[glossary:rule of dogma]{\textbf{rule of dogma}}.}\\
\rowlabel{glossary:faction} \\* \textbf{faction} & \parbox[t]{\linewidth}{A player; i.e. "this stone's faction" synonymous to "the player this stone belongs to".}\\
\rowlabel{glossary:flag} \\* \textbf{flag} & \parbox[t]{\linewidth}{A command placed at a specific position, for a specific stone, by the player. Flags can order the stones to attack, move, time-jump etc.}\\
\rowlabel{glossary:interference} \\* \textbf{interference} & \parbox[t]{\linewidth}{The act of obstructing the trajectory of a stone established in previous rounds. This makes the stone causally free again, but can also prevent it from activating its associated retro-cases.}\\
\rowlabel{glossary:opposable} \\* \textbf{opposable} & \parbox[t]{\linewidth}{If a stone is opposable, then it moves in a way which allows it to push other stones, but its movement can also be blocked by a stone moving in the opposite direction. See \hyperref[glossary:unopposable]{\textbf{unopposable}}.}\\
\rowlabel{glossary:orientable} \\* \textbf{orientable} & \parbox[t]{\linewidth}{If a stone is orientable, then its trajectory is described not only by its positions, but also their corresponding azimuths. Orientable stones move and attack in ways which depend on their azimuth. See \hyperref[glossary:azimuth]{\textbf{azimuth}}.}\\
\rowlabel{glossary:paradox} \\* \textbf{paradox} & \parbox[t]{\linewidth}{A scenario such that if we evolve the board along the time axis, there will either be active ante-effects which have zero or multiple activated retro-causes, or activated retro-causes which correspond to an inactive ante-effect.}\\
\rowlabel{glossary:position} \\* \textbf{position} & \parbox[t]{\linewidth}{One square on the board in one specific time-slice. Parametrised by three numbers: (t, x, y). See \hyperref[glossary:spatial position]{\textbf{spatial position}}, \hyperref[glossary:causally free]{\textbf{causally free}}.}\\
\rowlabel{glossary:precede} \\* \textbf{precede} & \parbox[t]{\linewidth}{Occur in a time-slice with time smaller than}\\
\rowlabel{glossary:priority} \\* \textbf{priority} & \parbox[t]{\linewidth}{Every scenario has a priority. If two or more scenarios are causally consistent for a round, the one with highest priority is canonised. See \hyperref[glossary:scenario]{\textbf{scenario}}, \hyperref[glossary:canonisation]{\textbf{canonisation}}.}\\
\rowlabel{glossary:progenitor} \\* \textbf{progenitor} & \parbox[t]{\linewidth}{A progenitor is a flag which, when active, places a new stone onto its associated position. Progenitors are unique in the way that they are associated with a specific stone, but do not require the presence of that stone at their position to be activated--rather, when activated, they place the stone themselves.}\\
\rowlabel{glossary:retro-cause} \\* \textbf{retro-cause} & \parbox[t]{\linewidth}{A cause that occurs after its effect on the time-axis. We say that a retro-cause is \textit{activated} if it is executed by its corresponding stone at its position in some specific scenario. See \hyperref[glossary:ante-effect]{\textbf{ante-effect}}.}\\
\rowlabel{glossary:round} \\* \textbf{round} & \parbox[t]{\linewidth}{The progress of the game occurs in rounds. In each round, players command their stones for every time-slice with increasing time, and after reaching the end, the round is canonised. See \hyperref[glossary:turn]{\textbf{turn}}, \hyperref[glossary:canonisation]{\textbf{canonisation}}.}\\
\rowlabel{glossary:rule of dogma} \\* \textbf{rule of dogma} & \parbox[t]{\linewidth}{All ante-effects placed in this round are dogmatic in the next round. See \hyperref[glossary:dogma]{\textbf{dogma}}.}\\
\rowlabel{glossary:scenario} \\* \textbf{scenario} & \parbox[t]{\linewidth}{A scenario for a given round specifies which ante-effects placed in all the previous round are active and which are passive in that round, and which setup placements are omitted. See \hyperref[glossary:activity]{\textbf{activity}}, \hyperref[glossary:setup]{\textbf{setup}}.}\\
\rowlabel{glossary:setup} \\* \textbf{setup} & \parbox[t]{\linewidth}{A set of progenitors which place stones on the board in the first time-slice. See \hyperref[glossary:progenitor]{\textbf{progenitor}}.}\\
\rowlabel{glossary:spatial position} \\* \textbf{spatial position} & \parbox[t]{\linewidth}{One square on the board across all time-slices. Parametrised by two numbers: (x, y). See \hyperref[glossary:position]{\textbf{position}}.}\\
\rowlabel{glossary:stone} \\* \textbf{stone} & \parbox[t]{\linewidth}{A stone is a unit which belongs to a player and is commanded by them, or is neutral (for example a box or a mine). The term \textbf{stone} implies continuity across time-slices but not across time-jumps. In other words, if a stone is placed on the board in the first time-slice and then makes a series of spatial moves, which propagate it into the final time-slice, we say it is still the same stone; however, if that stone then jumps back in time into the first time-slice, we say that the stone placed in the first time-slice is another, new stone. See \hyperref[glossary:worldline]{\textbf{worldline}}.}\\
\rowlabel{glossary:swap} \\* \textbf{swap} & \parbox[t]{\linewidth}{To swap an ante-effect means to add a new retro-cause to it, which can prevent its deactivation during subsequent canonisations, but can also result in a paradox if the number of retro-causes becomes too big. See \hyperref[glossary:paradox]{\textbf{paradox}}.}\\
\rowlabel{glossary:time} \\* \textbf{time} & \parbox[t]{\linewidth}{An axis on the board, akin to horizontal and vertical position.}\\
\rowlabel{glossary:time cap} \\* \textbf{time cap} & \parbox[t]{\linewidth}{The length of the time axis. The value of time runs from $0$ to $(\text{time cap}) - 1$.}\\
\rowlabel{glossary:time-jump} \\* \textbf{time-jump} & \parbox[t]{\linewidth}{The act of jumping back in time. For every stone, there will be opportunity to make a time-jump and re-join the game in the next round, when the earlier time-slices become available to place flags in once again.}\\
\rowlabel{glossary:time-slice} \\* \textbf{time-slice} & \parbox[t]{\linewidth}{A section of the board specified by a given value of time. Analogous to "rank" and "file" for horizontal and vertical position in chess.}\\
\rowlabel{glossary:trajectory} \\* \textbf{trajectory} & \parbox[t]{\linewidth}{The set of positions along the time axis of a stone from the time when it was placed on the board until the time it was removed from the board, or became causally free.}\\
\rowlabel{glossary:turn} \\* \textbf{turn} & \parbox[t]{\linewidth}{Every round is divided into a number of turns equal to the time cap. Each turn corresponds to one time-slice (the \textbf{active} time-slice), and in each turn every player places down flags for their causally free stones in that time-slice. See \hyperref[glossary:round]{\textbf{round}}, \hyperref[glossary:active]{\textbf{active}}.}\\
\rowlabel{glossary:unopposable} \\* \textbf{unopposable} & \parbox[t]{\linewidth}{If a stone is unopposable, then it essentially jumps over other stones, and thus cannot be blocked by a stone moving in the opposite direction. However, such a stone's movement is always blocked when it attempts to jump onto an occupied square, since it is unable to push other stones. See \hyperref[glossary:opposable]{\textbf{opposable}}.}\\
\rowlabel{glossary:unorientable} \\* \textbf{unorientable} & \parbox[t]{\linewidth}{If a stone is unorientable, then its trajectory is described only by its positions. The movement and attack pattern of unorientable stones is symmetric under rotation by a quarter-revolution. See \hyperref[glossary:orientable]{\textbf{orientable}}.}\\
\rowlabel{glossary:worldline} \\* \textbf{worldline} & \parbox[t]{\linewidth}{A worldline is a collection of trajectories belonging to a sequence of stones where each stone ends their trajectory by a time-jump which creates the next stone. Unlike the total number of stones on the board, the total number of worldlines is limited for the entire game by the number of stones placed on setup. A worldline can change its structure as the game progresses, by interference and swapping. See \hyperref[glossary:trajectory]{\textbf{trajectory}}, \hyperref[glossary:swap]{\textbf{swap}}.}


\end{longtable}}