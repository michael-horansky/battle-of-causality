\documentclass[12pt]{article}
\usepackage[a4paper, total={6.5in, 8in}]{geometry}
\usepackage{xargs}
\usepackage{amsmath,amssymb}
\usepackage{physics}

\newtheorem{definition}{Definition}

\begin{document}

	\title{Counting Novikov worldlines on potential paternity graphs}
	\author{Michal Horanský}
	\maketitle
	
	\section{Introduction}
	
	\textit{Note}: $P_{a\to b}$ here denoted a simple directed path from vertex $a$ to vertex $b$.
	
	\begin{definition}
	A \textbf{potential paternity graph} $G$ is a directed graph with vertices $V$ such that
	
	$$\exists V_0 \subset V: (\forall a \in V, \exists b \in V_0: \exists P_{b\to a} \qq{and} \forall c \in V_0, \nexists d \in (V-V_0): \exists P_{d \to c})$$
	
	Simply put, every vertex in $G$ can be reached from $V_0$, and no arc points at $V_0$.
	\end{definition}
	
	Note that $G$ is not required to be connected or simple.
	
	\begin{definition}
	A \textbf{worldline} is a simple directed path $P_{a\to b}$ on $G$ such that
	\begin{enumerate}
	\item either $a \in V_0$
	\item or there exists an arc $b\to a$, and the vertices of $P_{a\to b}$ therefore can form a directed cycle.
	\end{enumerate}
	\end{definition}
	In the first case, the worldline is said to be \textbf{rooted}; in the second case, it is said to be \textbf{looped}.
	
	\begin{definition}
	A \textbf{realisation} of $G$ is a subset of its vertices, $R \subset V$, such that $R$ can be partitioned into worldlines.
	
	The empty set is called the \textbf{trivial realisation} of $G$.
	\end{definition}
	
	Example: consider the potential paternity graph $A$ formed by $N+1$ vertices $a_0\dots a_N$ such that there is an arc $a_i \to a_{i+1}$ for $i=0 \dots N-1$, and $V_0 = \{a_0\}$. Then, each worldline has the shape $\{a_0, a_1 \dots a_m\}$ for $m \in \{0 \dots N\}$. Therefore, there are $N+1$ possible worldlines in $A$, and therefore $N+2$ valid realisations (since no two worldlines are disjoint), also counting the trivial realisation.
	
	Even though the partitioning of a realisation into worldlines may not be unique, the number of rooted worldlines is uniquely specified for a realisation (but not the number of looped worldlines: for an example, consider a complete directed graph with a looped worldline of more than one vertex: this worldline may be further partitioned into smaller looped worldlines). The number of rooted worldlines in a realisation is trivially equal to the number of vertices in the realisation which belong into $V_0$: this is called the \textbf{inclusion number} of the realisation.
	
	\section{Problems}
	
	The following are questions regarding potential paternity graphs which are the subjects of this inquiry.
	
	\subsection{Counting realisations}
	Given a potential paternity graph $G$, how many realisations of $G$ are there?
	
	\subsection{Iterating through realisations}
	We wish to find a deterministic algorithm which iterates through the set of all realisations of $G$.
	
	\subsection{Iterating through realisations with preferentiality} \label{sec:preferential iteration}
	Given a preference $P$ which orders the set of realisations, we wish to find a deterministic algorithm which iterates through the set of all realisations of $G$ (denoted $\mathcal{R}$) in order given by $P$. There are restrictions on the form of $P$.
	
	Typically, $P$ can be expressed as a sequence of preferences $(P_1, P_2 \dots P_\pi)$ such that each each $P_i$ partially orders $\mathcal{R}$. Then, $P$ applies to $\mathcal{R}$ like so: for two realisations $a, b\in \mathcal{R}$, if $P_1$ orders $a,b$, that order is applied; if not, $P_2$ is checked and so on. It other words, $(P_i)$ defines a \textit{lexicographical} ordering. Note that $P$ must impose a total order.
	
	The form of $P_i$ can be chosen from the following categories:
	\begin{itemize}
	\item An ordering is chosen for the vertices $V$ of $G$. For two realisations $a,b$, the first vertex according to the vertex ordering which is included in one but not both realisations is considered: the realisation which includes this vertex preceded the realisation which does not. (This is also a lexicographical order.)
	\item Same as above, but only ordering vertices in $V_0$ of $G$: inclusion of the first vertex in $V_0$ which is not included in both $a$ and $b$ will order them.
	\item For two realisations $a,b$, the realisation with a higher number of vertices precedes the other one.
	\item For two realisations $a,b$, the realisation with a higher inclusion number precedes the other one.
	\end{itemize}
	
	Note that the first of these always imposes a total order, and thus its inclusion in the preference sequence is sufficient for $P$ to be a total order; the sequence also always terminates at that point, since no subsequent (lower priority) preference can affect $P$.
	
	\subsection{Realisation inclusion check}
	This is only an interesting question if the question in Sec. \ref{sec:preferential iteration} proves to be too difficult to be answered. For a given subset of vertices $v\in V$ we wish to determine whether $v$ is a realisation of the potential paternity graph. What is the fastest way to do this?
	
\end{document}